\documentclass{bioinfo}
\usepackage[english]{babel}
\copyrightyear{2015} \pubyear{2015}
\usepackage{natbib}
\bibliographystyle{natbib.bst}


\access{Advance Access Publication Date: Day Month Year}
\appnotes{Manuscript Category}

\begin{document}
\firstpage{1}

\subtitle{Subject Section}

\title[short Title]{Combining Bottom-up and top-down approaches for drug-target-interaction prediction}
\author[Sample \textit{et~al}.]{Tilman Hinnerichs\,$^{\text{\sfb 1,}*}$ and Robert Hoehndorf\,$^{\text{\sfb 2}}$}
\address{$^{\text{\sf 1}}$Department, Institution, City, Post Code, Country and \\
$^{\text{\sf 2}}$Department, Institution, City, Post Code,
Country.}

\corresp{$^\ast$To whom correspondence should be addressed.}

\history{Received on XXXXX; revised on XXXXX; accepted on XXXXX}

\editor{Associate Editor: XXXXXXX}

\abstract{\textbf{Motivation:} Text Text Text Text Text Text Text Text Text Text Text Text Text
Text Text Text Text Text Text Text Text Text Text Text Text Text Text Text Text Text Text Text
Text Text Text Text Text Text Text Text Text Text Text Text Text Text Text Text Text Text Text
Text Text Text Text Text Text
Text Text Text Text Text.\\
\textbf{Results:} Text  Text Text Text Text Text Text Text Text Text  Text Text Text Text Text
Text Text Text Text Text Text Text Text Text Text Text Text Text  Text Text Text Text Text Text\\
\textbf{Availability:} Text  Text Text Text Text Text Text Text Text Text  Text Text Text Text
Text Text Text Text Text Text Text Text Text Text Text Text Text Text  Text\\
\textbf{Contact:} \href{tilman.hinnerichs@kaust.edu.sa}{tilman.hinnerichs@kaust.edu.sa}\\
\textbf{Supplementary information:}10264703 Supplementary data are available at \textit{Bioinformatics}
online.}

\maketitle

\section{Introduction}

\cite{Survey2018}\\
In history, traditional remedies, that were known for their medicinal properties lead to drugs by extraction of the functional ingredients. Alternatively, characteristics and features of potential drugs were detected by accident like in the case of penicillin. More recently, biological drug targets can be found \textit{in silico} through discovery of suitable computational predictors.\\
The challenge of accurately predicting drug-target-interactions (DTI) has shown its importance in the fields of drug repurposing and repositioning, and in the exploration of novel drugs and their interaction partners. Knowledge about those links between compounds and their target proteins help in an array of medical and pharmaceutical studies. Additionally, those associations can be utilized to identify disease specific targets, leading to desirable therapeutic effects.\\
With the rapidly growing field of machine learning approaches and their application to bioscientifical problems in the realm of bioinformatics, different kinds of data, such as long DNA sequences could be utilized for feature generation, while rapid advances were made. Almost all state of the art models for drug-target-interaction prediction were based on the usage of neural networks with increasing size.\\
Only recently, the technique of graph learning was introduced by \citet{GCNConv} through graph convolution algorithms, and improved and altered under usage of different kernels(\cite{ChebConv}, \cite{ARMAConv}) attention mechanisms (\cite{GATConv}), random walks (\cite{APPNPConv}) and mixtures of both (\cite{SAGEConv}). While based on diverse systems, they can be relevant for testing distinct hypothesis for given graphs. While convolutional filters are suitable for finding patterns among the the given graph, attention mechanisms are more relevant for discovery of important regions within. Lately, graph learning approaches found application for computing compound representations for DTI prediction.\\
Approaches on this rather sophisticated problem can divided into top-down or network approaches (\textbf{CITATION}), and bottom-up or molecular approaches (\textbf{CITATION}). Top-down approaches take advantage of other data such as diseases (CITATION), side effects, knowledge graphs or ontologies, in order to learn representations for both compound and protein. On the other hand, bottom-up aspirations attempt to learn from chemical properties. For drugs, molecular structure (CITATION GraphDTA), molecular fingerprints, similarity to other drugs (See Bioinf Survey), and other molecular features. On the protein side, secondary structure prediction (CITATION), contact prediction (CITATION), or simply convolution over the amino acid sequences served to obtain a feature for the given proteins. However, both kinds contain and share some problems, that are not solvable within themselves. Thus, bottom-up approaches share the lack of ability to generalize, which we will show in later sections, and usually focus on engineering sophisticated features for the drugs, while neglecting to formulate meaningful features on the protein side. Top-down approaches lack the ability to spot small differences to cope with small differences within the drug structure and rely heavily on given data for the considered drug-target pair. The latter is not suitable for predictions on novel or unseen compounds, as e.g., data on side effects or its impact on diseases is seldom given for novel drugs.\\ 
In order to design such a feature for proteins and drugs, respectively, we make use of the interaction networks for both proteins and compounds. Drug-drug interaction networks were introduced and standardized by \cite{Boyce2015} and have been used for clinical decision support (\cite{Scheife2015}). Drug-drug interaction networks may give a hint on common targeted pathways. As an additional compound feature we will use semantic side effect similarity, which we will discuss later on.\\
Protein-protein interaction networks have shown great results in $\dots$(\cite{Vazquez2003}, \cite{Ackerman2019}) in granting context for molecular system biology. However, these contexts were never applied to the problem of drug-target-interaction prediction. Thus we formalized our hypotheses over these interaction graphs and will test them in the following chapters.

%\enlargethispage{12pt}

\section{Approach}
\subsection{Choosing appropriate train-test splits}
As shown, recent work lacks the ability to combine both top-down and bottom-up approaches for their predictions, however performing quite well on their datasets. Drug-target interaction prediction is the task to accurately predict whether for a given drug and a given protein there is a biological interaction within humans. However, when building the train-test split over compound-protein pairs for building the actual model, there are the following three options:
\begin{enumerate}
	\item Build split over drugs
	\item Build split over drug-target pairs
	\item Build split over proteins
\end{enumerate} 
In general, recent works do perform their split over the drugs or drug-target pairs (\cite{Survey2018}, CITATION). The first is more relevant for novel drugs, as it is much more likely to test a new compound than a innovative protein. However, it lies in the very nature of the used datasets, making the prediction for new drugs much easier. Thus, drugs are often built by minor variations of existing drugs, thus leading to no deviations in the functional group of that very compound (CITATION/EXAMPLE). When distributed over both train and test split, the models do not perform inductive inference and generalize, but rather implement transductive inference by just predicting the recently seen structures. Hence, when entirely new molecules are seen, the models perform much worse. \\
The same applies to splits of drug-target pairs, as all drugs were already seen, and novelty cannot be coped with.\\
As mentioned in the introduction it is quite difficult to learn suitable features from proteins. In general, attempts search for motifs in the protein sequences under usage of convolutional neural networks and filters, which is more suitable for tasks like protein function prediction, than for for drug-target interaction prediction, and lack a more in-depth hypothesis on the protein side, while investing in refined drug features. \\
Thus, building splitting over proteins is the most challenging of the three options. \\

\subsection{Tested hypotheses}

In this work we are testing the following hypotheses:
\begin{enumerate}
	\item Can we build a model that outperforms state of the art approaches, combining top-down and bottom-up approaches?
	\item Are interaction networks sufficient to improve the performance of simple molecular predictors?
\end{enumerate}
We will test the first hypothesis by building a model that takes both top-down and bottom-up features into account. Thus, we propose a novel approach to combine those mutual exclusive attempts, through the usage of interaction networks, similarity and molecular features. Additionally, we test the latter by building a simple molecular DTI predictor and enhance it under usage of the interaction networks.\\

For the bottom-up approach we build a that only relies only on molecular features, which we will discuss in more detail in the following methods chapter. For the combination of both approaches we now attach the predictions to the protein-protein interaction graph as node features for future graph learning steps. In this graph we tried to find both patterns and regions for each drug that could be of interest through application of different graph convolutional layers, which in return represent the feature for each protein. Representing the drug we take the drug-drug interaction graph and the semantic similarity over side effects which we will explain in the following paragraphs.

\section{Methods}
\subsection{Data}
The data for the different parts of this model were obtained from different sources. Starting with the protein-protein interactions, we fetched 11.574 proteins with over 170.000 links from STRING (\cite{STRINGv10}). As STRING provides statistical scores for each association, we filtered them as advised by a threshold of 700. For the drug-drug interaction we retrieved the drugs from \cite{Boyce2015}. Side effects were obtained from MedDRA database (\cite{MedDRA}) that hierarchically orders side effects. Links between drugs and side effects were fetched from \textit{SIDER} (\cite{SIDER}) side effects database. The intersection between these sources yielded 641 common drugs for usage in our data. For the drug-target interactions themselves, we fetched 
137.000 links from STITCH database (\cite{STITCHv5}). We provide links and methods to and for the necessary data in the provided Github repository.\\

\subsection{Models}
The used model consists of two separate models, that help to fuse together the two methods:
\begin{enumerate}
	\item The molecular predictor
	\item The interaction network based predictor
\end{enumerate}
We build the molecular predictor by using pretrained, molecular fingerprints models for both drugs and proteins. Regarding proteins, we used the pretrained feature generator from \textit{DeepGoPlus} (\citep{DeepGoPlus}) that was originally designed for protein function prediction and is regarded as state of the art for this purpose. For drugs we used a pretrained fingerprint model from SMILES transformer (\cite{SmilesTransformer}), that provides a simple and fast method to compute fingerprints through autoencoder models. The encodings from these two models were funneled into a simple deep neural network (TABLE) with few fully connected. \\
The results of that prediction flow into the annotation of the protein-protein interaction (PPI) graph as depicted in (IMAGE). Hereby, the predictions of the molecular predictor are used as node features for the graph, with respect to the given drug. Thus, given a compound-target pair, the nodes of the PPI graph now hold bottom-up features, which can now be processed by the graph algorithms. \\
The PPI graph is processed by different graph convolutional layers, that may underline the importance of either patterns or regions within the graph, to obtain a feature vector for the wanted node. In contrast to learning over whole graphs we perform node classification within the graph. \\
The drug-drug interaction features are retrieved by choosing the corresponding row in the adjacency matrix of the graph, thus leading to quite simple features. \\
For the semantic similarity feature, that once again represents a top-down attribute, we artificially link each drug to its corresponding side effects in the MedDRA hierarchy. Concerning this hierarchy, drug-drug similarity is computed by the Resnik similarity (\cite{Resnik1995}). For the given compound we take the corresponding row of this symmetric similarity matrix. \\

Thereby, we concatenate these three features together and funnel them into another deep neural network. 

\section{Discussion}







%%%%%%%%%%%%%%%%%%%%%%%%%%%%%%%%%%%%%%%%%%%%%%%%%%%%%%%%%%%%%%%%%%%%%%%%%%%%%%%%%%%%%
%
%     please remove the " % " symbol from \centerline{\includegraphics{fig01.eps}}
%     as it may ignore the figures.
%
%%%%%%%%%%%%%%%%%%%%%%%%%%%%%%%%%%%%%%%%%%%%%%%%%%%%%%%%%%%%%%%%%%%%%%%%%%%%%%%%%%%%%%






\section{Conclusion}

\vspace*{-10pt}


\section*{Acknowledgements}

\vspace*{-12pt}

\section*{Funding}

This work has been supported by the... Text Text  Text Text.\vspace*{-12pt}

%\bibliographystyle{natbib}
%\bibliographystyle{achemnat}
%\bibliographystyle{plainnat}
%\bibliographystyle{abbrv}
%\bibliographystyle{bioinformatics}
%
%\bibliographystyle{plain}
%
\bibliography{citations}


\end{document}
