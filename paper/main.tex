\documentclass{bioinfo}
\usepackage[english]{babel}
\copyrightyear{2015} \pubyear{2015}
\bibliographystyle{plainnat}

\usepackage{natbib}

\access{Advance Access Publication Date: Day Month Year}
\appnotes{Manuscript Category}

\begin{document}
\firstpage{1}

\subtitle{Subject Section}

\title[short Title]{Combining Bottom-up and top-down approaches for drug-target-interaction prediction}
\author[Sample \textit{et~al}.]{Tilman Hinnerichs\,$^{\text{\sfb 1,}*}$ and Robert Hoehndorf\,$^{\text{\sfb 2}}$}
\address{$^{\text{\sf 1}}$Department, Institution, City, Post Code, Country and \\
$^{\text{\sf 2}}$Department, Institution, City, Post Code,
Country.}

\corresp{$^\ast$To whom correspondence should be addressed.}

\history{Received on XXXXX; revised on XXXXX; accepted on XXXXX}

\editor{Associate Editor: XXXXXXX}

\abstract{\textbf{Motivation:} Text Text Text Text Text Text Text Text Text Text Text Text Text
Text Text Text Text Text Text Text Text Text Text Text Text Text Text Text Text Text Text Text
Text Text Text Text Text Text Text Text Text Text Text Text Text Text Text Text Text Text Text
Text Text Text Text Text Text
Text Text Text Text Text.\\
\textbf{Results:} Text  Text Text Text Text Text Text Text Text Text  Text Text Text Text Text
Text Text Text Text Text Text Text Text Text Text Text Text Text  Text Text Text Text Text Text\\
\textbf{Availability:} Text  Text Text Text Text Text Text Text Text Text  Text Text Text Text
Text Text Text Text Text Text Text Text Text Text Text Text Text Text  Text\\
\textbf{Contact:} \href{tilman.hinnerichs@kaust.edu.sa}{tilman.hinnerichs@kaust.edu.sa}\\
\textbf{Supplementary information:}10264703 Supplementary data are available at \textit{Bioinformatics}
online.}

\maketitle

\section{Introduction}

\cite{Survey2018}\\
In history, traditional remedies, that were known for their medicinal properties lead to drugs by extraction of the functional ingredients. Alternatively, characteristics and features of potential drugs were detected by accident. More recently, biological drug targets were found \textit{in silico} through discovery of suitable computational predictors.\\
The challenge of accurately predicting drug-target-interactions (DTI) has shown its importance in the fields of drug repurposing and repositioning, and in the exploration of novel drugs and their interaction partners. Knowledge about those links between compounds and their target proteins help in an array of medical and pharmaceutical studies. Additionally, those associations can be utilized to identify disease specific targets, leading to desirable therapeutic effects.\\
With the rapidly growing field of various machine learning approaches and their application to bioscientifical problems in the realm of bioinformatics, rapid advances were made. Almost all state of the art models for drug-target-interaction prediction were based on the usage of neural networks with increasing size.\\
Approaches on this rather sophisticated problem are either based on top-down or network approaches (\textbf{CITATION}), or bottom-up and molecular approaches (\textbf{CITATION}). However, both kinds contain some problems, that are not solvable within their own system. Thus, bottom-up approaches share the lack of ability to generalize, which we will show in later sections. This 

%\enlargethispage{12pt}

\section{Approach}


\section{Methods}



\section{Discussion}







%%%%%%%%%%%%%%%%%%%%%%%%%%%%%%%%%%%%%%%%%%%%%%%%%%%%%%%%%%%%%%%%%%%%%%%%%%%%%%%%%%%%%
%
%     please remove the " % " symbol from \centerline{\includegraphics{fig01.eps}}
%     as it may ignore the figures.
%
%%%%%%%%%%%%%%%%%%%%%%%%%%%%%%%%%%%%%%%%%%%%%%%%%%%%%%%%%%%%%%%%%%%%%%%%%%%%%%%%%%%%%%






\section{Conclusion}

\vspace*{-10pt}


\section*{Acknowledgements}

\vspace*{-12pt}

\section*{Funding}

This work has been supported by the... Text Text  Text Text.\vspace*{-12pt}

%\bibliographystyle{natbib}
%\bibliographystyle{achemnat}
%\bibliographystyle{plainnat}
%\bibliographystyle{abbrv}
%\bibliographystyle{bioinformatics}
%
%\bibliographystyle{plain}
%
\bibliography{citations}


\end{document}
