\documentclass[]{article}
\usepackage[left=3cm,right=3cm,top=1.5cm,bottom=2cm,includeheadfoot]{geometry} 
\usepackage{babel}
\usepackage{hyperref}
\usepackage{mwe}
\usepackage[markcase=noupper
]{scrlayer-scrpage}
\usepackage{amsmath}
\usepackage{amssymb}

\KOMAoptions{
	headsepline = true
}
\ihead{Tilman Hinnerichs}
\ohead{Combining bottom-up and top-down approaches}
\cfoot*{\pagemark}
%opening
\title{ECCB paper: Combining bottom-up and top-down approaches for drug-target-interaction prediction}
\author{Tilman Hinnerichs\\
\href{mailto:tilman@hinnerichs.com}{tilman@hinnerichs.com}}
\date{}
\pagestyle{headings}

% Build subsubsubsection
\usepackage{titlesec}

\titleclass{\subsubsubsection}{straight}[\subsection]

\newcounter{subsubsubsection}[subsubsection]
\renewcommand\thesubsubsubsection{\thesubsubsection.\arabic{subsubsubsection}}
\renewcommand\theparagraph{\thesubsubsubsection.\arabic{paragraph}} % optional; useful if paragraphs are to be numbered

\titleformat{\subsubsubsection}
{\normalfont\normalsize\bfseries}{\thesubsubsubsection}{1em}{}
\titlespacing*{\subsubsubsection}
{0pt}{3.25ex plus 1ex minus .2ex}{1.5ex plus .2ex}

\makeatletter
\renewcommand\paragraph{\@startsection{paragraph}{5}{\z@}%
	{3.25ex \@plus1ex \@minus.2ex}%
	{-1em}%
	{\normalfont\normalsize\bfseries}}
\renewcommand\subparagraph{\@startsection{subparagraph}{6}{\parindent}%
	{3.25ex \@plus1ex \@minus .2ex}%
	{-1em}%
	{\normalfont\normalsize\bfseries}}
\def\toclevel@subsubsubsection{4}
\def\toclevel@paragraph{5}
\def\toclevel@paragraph{6}
\def\l@subsubsubsection{\@dottedtocline{4}{7em}{4em}}
\def\l@paragraph{\@dottedtocline{5}{10em}{5em}}
\def\l@subparagraph{\@dottedtocline{6}{14em}{6em}}
\makeatother

\setcounter{secnumdepth}{4}
\setcounter{tocdepth}{4}

%\usepackage{fancyhdr}

\newcommand{\HRule}[1]{\rule{\linewidth}{#1}}

% Algorithm
\usepackage{algorithm}
\usepackage[noend]{algpseudocode}
\usepackage{amsmath}
\def\BState{\State\hskip-\ALG@thistlm}
\makeatother

\begin{document}
	
	\title{ \normalsize \textsc{MPI: MiS}
		\\ [0.5cm]
		\HRule{0.5pt} \\
		\LARGE \textbf{\uppercase{Combining bottom-up and top-down approaches for drug-target-interaction prediction}}
		\HRule{0.5pt} \\ [0.5cm]
		\normalsize }

\maketitle


\section{Recent work}
	\begin{itemize}
		\item Write about how to find suitable representations for both drugs and proteins
		\item Recent papers (see survey paper from briefings in Bioinformatics) $\rightarrow$ use either bottom up or top down
		\item best bottom up and top down results come from which papers?
		\item What are problems of recent approaches?
		\begin{itemize}
			\item Lack ability to generalize (bottom-up) or are unable to spot small differences (top-down)
			\item only top down or bottom-up
			\item not making use of interaction networks
		\end{itemize}
		\item They all do split over drugs and not over proteins $\rightarrow$ why is that bad? How can that be omitted? What consequences does that have for new drug-target pairs? 
		\begin{itemize}
			\item its feasible to engineer sophisticated representations for drugs, but not really for the proteins, both in bottom-up and in top-down approaches
			\item Thus, we will focus on building proper representations for the proteins
		\end{itemize}
		\item Additionally, PPI-graphs have been used in recent works to do \dots, but have found no application in dti prediction. This also applies for DDI-graphs With more and more uprising Graph Learning approaches, we can learn more sophisticated representations and test more complicated hypothesis.
		
	\end{itemize}

\section{How to combine mutual exclusive approaches}
\subsection{Problem description}
	\begin{itemize}
		\item What is dti prediction? $\rightarrow$ problem description
		\item Emphasize the importance of ability to generalize
		\item Issues with data in general $\rightarrow$ Open vs. closed world assumption. (To specific? Can we address this issue?)
		\item 
	\end{itemize}
\subsection{Issues to be solved}
	\begin{itemize}
		\item We are testing two hypothesis at the same time:
			\begin{itemize}
				\item Can we use PPI and DDI graphs to learn a protein representation that generalizes better?
				\item Can the built filter increase the performance of that very method? (Find better formulation)
			\end{itemize}
	\end{itemize}
\subsection{Role of interaction networks}
\subsubsection{DDI}
	\begin{itemize}
		\item What are DDI networks? What are they able to tell?
		\item Why are they important?
		\item Where do they come from? $\rightarrow$ Boyce et al., Maybe repeat some ideas from their paper
	\end{itemize}
\subsubsection{PPI}
	\begin{itemize}
		\item What are PPI networks? What can they tell? What does a link in the graph mean? Why would that be important to our problem? (Role of pathways)
		\item Recent work on PPIs? Add some ideas from them.
		\item What are we actually trying to find? (Regions within PPI network that are of interest for drug, \textbf{OR} patterns in the PPI graph that are of interest for the drug)
	\end{itemize}
\subsection{Finding node features for the PPI graph}
\subsubsection{Encoding of number of neighbours}
	\begin{itemize}
		\item Encode number of neighbours in node features $\rightarrow$ Many neighbours $\implies$ well studied and possibly many targets
	\end{itemize}
\subsubsection{How to build a drug specific filter (optional)}
	\begin{itemize}
		\item STITCH got 390.000 chemicals, 3.6 million proteins from over 2000 organisms
		\item for each drug find a motif in the target amino acid squences and query against all proteins and use the results as features for the prediction
		\item Why do need those filters? Why should they work (Discuss with Robert)? How do they help our prediction? 
		\item write down algorithm that is used: alignment $\rightarrow$ HMM $\rightarrow$ query
	\end{itemize}
\subsection{Features for drugs}
	Similarity scores over MedDRA data. 
	\begin{itemize}
		\item Why can they be useful? How did we obtain them?
	\end{itemize}

\section{Models}
	Transductive vs inductive models
	\begin{itemize}
		\item When is what kind of model better for the model? For which one of the two hypothesis? (transductive for patterns and inductive for regions of interest)
		\item What architectures are used? (Plain convolution for patterns and Attention based fo regions)
		\item Maybe some images of model structure
	\end{itemize}

\section{Experiments}
None yet

\section{Discussion}

\section{Future Work}




\end{document}